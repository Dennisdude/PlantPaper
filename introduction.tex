\section{Einleitung}
   Obwohl Pflanzen kein ausgeprägtes Gehörorgan besitzen und Geräusche im her- kömmlichen Sinne nicht wahrnehmen können, haben sie dennoch erstaunliche Fähig- keiten zur Wahrnehmung ihrer Umgebung. Sie reagieren auf Schallwellen indirekt durch Vibrationen, die sie über verschiedene Mechanismen erfassen können. Die Studie hat gezeigt, dass Pflanzen auf eine Vielzahl von Schwingungen reagieren, sei es durch Schallwellen, Windbewegungen oder Berührungen. Diese Fähigkeit ermöglicht es ihnen, ihre Umgebung zu erkennen und angemessen darauf zu reagieren, indem sie beispielsweise ihr Wachstum, ihre Blattstellung oder ihre Abwehrmechanismen anpassen \cite{bachmann-wampfler2022}. \\
   
   Es wurde gezeigt, dass Musik einen Einfluss auf das Wachstum und die Entwicklung von Pflanzen haben kann. Verschiedene Musikstile und Frequenzen scheinen unterschiedliche Auswirkungen auf Parameter wie Pflanzenhöhe, Blattfläche, Wurzel- wachstum und Blütenbildung zu haben \cite{chowdhury-gupta2015}. \\
   
   Die zugrundeliegenden Mechanismen dieser Wechselwirkung zwischen Musik und Pflanzenwachstum sind noch nicht vollständig verstanden. Einige Forschende vermuten, dass Pflanzen auf die Schwingungen und Frequenzen der Musik reagieren, die möglicherweise Einfluss auf zelluläre Prozesse und Signalwege nehmen \cite{gagliano2013}. \\
   
   In der vorliegenden Studie untersuchen wir die Auswirkungen verschiedener verbaler Signale auf das Wachstum und die Entwicklung von Samen der Ringelblume 'Candyman Orange' von Thompson Morgan. Unter verbalen Signalen verstehen wir sowohl positiven Zuspruch (Lob) als auch negative Äußerungen (Beleidigungen). Lob könnte beispielsweise freundliche, ermutigende Wörter wie "wunderschön", "toll gemacht" oder "beeindruckend" sein, während Beleidigungen abwertende, verletzende Äußerungen wie "hässlich", "schlecht" oder "minderwertig" wären. \\
    
    Wir möchten herausfinden, ob und in welchem Maße unterschiedliche verbale Darbietungen das Wachstum und die Blütenbildung dieser Ringelblumen-Art be-einflussen können. \\

    Diese Erkenntnisse können potenziell zur Optimierung von Anbaumethoden in der Landwirtschaft beitragen, indem sie neue Wege zur Steigerung des Pflanzenwachstums aufzeigen.