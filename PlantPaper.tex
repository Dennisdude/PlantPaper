\documentclass[11pt]{article}
\usepackage{hyperref}
\begin{document}
	\title{Der Einfluss verbaler emotionaler Beeinflussung auf das Wachstum und die Gesundheit von Pflanzen}
	\author{Alexander Schwab, Dennis Deifel, Rene Ott}
	\date{\today}
	\maketitle
	\tableofcontents

	\pagebreak

    \section{Einleitung}
    Zahlreiche Studien haben gezeigt, dass Musik einen Einfluss auf das Wachstum und die Entwicklung von Pflanzen haben kann. 
	Verschiedene Musikstile und -frequenzen scheinen unterschiedliche Auswirkungen auf Parameter wie Pflanzenhöhe, Blattfläche, 
	Wurzelwachstum und Blütenbildung zu haben. \\
	\\
    Die zugrunde liegenden Mechanismen dieser Wechselwirkung zwischen Musik und Pflanzenwachstum sind noch nicht vollständig 
	verstanden. Einige Forschende vermuten, dass Pflanzen auf die Schwingungen und Frequenzen der Musik reagieren, 
	die möglicherweise Einfluss auf zelluläre Prozesse und Signalwege nehmen. Andere Studien deuten darauf hin, 
	dass auditive Reize Pflanzen dazu anregen können, 
	ihre Ressourcen effizienter zu nutzen und so ihr Wachstum zu optimieren. \newline
	\newline
    In der vorliegenden Studie untersuchen wir die Auswirkungen verschiedener Musikstile auf das Wachstum und 
	die Entwicklung der Ringelblume 'Candyman Orange' von Thompson Morgan. Diese Pflanze wurde aufgrund ihrer Robustheit, 
	Farbintensität und Blühfreudigkeit ausgewählt. Wir möchten herausfinden, ob und in welchem Maße 
	unterschiedliche Musikdarbietungen das Wachstum und die Blütenbildung dieser Ringelblumenart beeinflussen können.
	\begin{itemize}
		\item \href{https://de.overleaf.com/learn}{https://de.overleaf.com/learn}
		\item \href{https://www.latex-project.org/help/documentation/}{https://www.latex-project.org/help/documentation}
	\end{itemize}

	\section{Wo bekomme ich wissenschaftliche Literatur her?}
	\subsection{Google Scholer}
	Google Scholer ist eine kostenlose Suchmaschine für wissenschaftliche Literatur. 
	Sie durchsucht wissenschaftliche Zeitschriften, Bücher und Konferenzbeiträge. 
	Die Suchergebnisse sind oft besser als bei der normalen Google-Suche, da Google Scholer nur wissenschaftliche Literatur 
	durchsucht. \cite{halevi2017suitability}

	\href{htts://scholar.google.de/}{https://scholar.google.de/}

	\subsection{Semantic Scholar}
	https://www.semanticscholar.org/

	\pagebreak

	\bibliography{mybib}{}
	\bibliographystyle{plain}
\end{document}