\section{Versuchsaufbau}
    \subsection{Allgemein}
    In einem kontrollierten Experiment werden Samen der \\ Ringelblume 'Candyman Orange' in drei Versuchsgruppen eingeteilt:
    \begin{enumerate}
        \item Kontrollgruppe: Keine verbalen Signale.
        \item Lobgruppe: Regelmäßige Zuwendung mit positiven, ermutigenden Äußerungen.
        \item Beleidigungsgruppe: Regelmäßige Zuwendung mit abwertenden, verletzenden Äußerungen.
    \end{enumerate} 
    
    Alle Gruppen erhalten identische Wachstumsbedingungen in Bezug auf Licht, Wasser, Temperatur und Bodenbeschaffenheit. Die Samen werden den natürlichen Temperatur-, Licht- und Luftfeuchtigkeitsbedingungen ausgesetzt und in gleich großen Töpfen mit derselben Blumenerdemischung platziert. \\

    \subsection{Auswahl der Pflanze}
    Verwendung der Ringelblume Candyman Orange von Thompson Morgan aufgrund ihrer Robustheit gegen Temperaturschwankungen und der schnellen Keim zeit von 1-2 Wochen.

    \subsection{Umfeld}
    Das Experiment findet in einer kontrollierten Umgebung statt, um ein konsistentes Umfeld für alle Samen sicherzustellen. Die Samen werden den natürlichen Umgebungsbedingungen des Raumes ausgesetzt, einschließlich der natürlichen Temperatur, Lichtintensität und Luftfeuchtigkeit. Die Samen werden in Töpfen mit gleicher Bodenmischung und Größe platziert.

    \subsection{Etablierung der Versuchspflanzen}
    \subsubsection{Kontrollpflanzen "Stabilitree"}
    Die Kontrollpflanzen Gruppe wurde als "Stabilitree" bezeichnet, um ihre Rolle als Referenzpunkt für die unbeeinflusste Entwicklung zu betonen. Die unter natürlichen Bedingungen wächst, ohne zusätzliche externe Einflüsse.
    
    \subsubsection{Lobpflanzen "Rootiful"}
    Die Pflanzen Gruppe, die lobende Worte erhält, wurde als "Rootiful" benannt. Hier werden freundliche, ermutigende Wörter wie "wunderschön", "toll gemacht" oder "beeindruckend" über den Lautsprecher abgespielt.
    
    \subsubsection{Beleidigungspflanzen "Mockstalk"}
    Die Pflanzen Gruppe, die beleidigende Äußerungen ausgesetzt ist, trägt den Namen "Mockstalk". Hier werden abwertende, verletzende Äußerungen wie "hässlich", "schlecht" oder "minderwertig" über den Lautsprecher abgespielt.

    \subsection{Beschallungsgerät}
    Jede Gruppe wird mit einem separaten Lautsprecher ausgestattet, wobei zwei \\ Creative MUVO Go Stereo-Lautsprecher verwendet werden, um die entsprechende auditive Stimulation zu liefern. Eine mittlere bis niedrige Lautstärke wird dabei für beide Pflanzen gleich eingestellt.

    \subsection{Datenerhebung und -analyse}
    Über einen Zeitraum von 21 Tagen werden folgende Parameter regelmäßig gemessen:
    \begin{itemize}
         \item Anzahl der Sprösslinge
         \item Größe der Sprösslinge
     \end{itemize}

    Nach dem Zeitraum werden die Keimlinge aus der Erde geholt und es werden folgende Parameter gemessen:
    \begin{itemize}
        \item Größe der Wurzeln
        \item Anzahl der Sprösslinge
        \item Größe der Sprösslinge
     \end{itemize}

    Bei der Bewertung wird die Anzahl jeweils höher gewichtet als die Größe und die Sprösslinge höher als die Wurzeln.
    
    \subsection{Kontrolle und Pflege}
    Alle Gruppen erhalten die gleiche Menge an Wasser in regelmäßigen Abständen, um optimale Bedingungen sicherzustellen. Jegliche Anomalien oder unerwartete Ereignisse werden dokumentiert und bei der Analyse berücksichtigt.